\documentclass[11pt]{article}
% \textwidth=6in
% \textheight=9in
% \oddsidemargin=0.25in
% \evensidemargin=0.25in
% \marginparwidth=0in
% \marginparsep=0in
% \topmargin=0in
% \headheight=0in
% \headsep=0in
% \topskip=0in
\usepackage{hyperref}
\usepackage{lscape}
\usepackage{longtable}
\usepackage{graphics}
\usepackage[pdftex]{graphicx}
\usepackage{color}
\usepackage{rotating}
\newcommand{\hilight}[1]{\colorbox{yellow}{#1}}

\newenvironment{packed_itemize}{
\begin{itemize}
  \setlength{\itemsep}{1pt}
  \setlength{\parskip}{0pt}
  \setlength{\parsep}{0pt}
}{\end{itemize}}

\title{CSCE 435 Parallel Computing, Fall 2017 \\
Project 1: parallel matrix multiplication}
\author{Tim Davis}
\date{Assigned Monday, Sept 25.  Due at 2pm on Tuesday, Oct 3}
\begin{document}

\maketitle

In this project you will write three matrix multiplication methods
and compare their performance:

\begin{itemize}
\item The first one is the simple method
in Problem 2.4 in the book.  This method will also be used 
to multiply each block in the second and third methods.

\item The second method is a sequential one, with no OpenMP parallelism,
but it computes its result \verb'C=A*B' by blocks, as in the solution to
Problem 2.5 from homework 1.

\item The third method is just like the second method, except that it
uses 20 threads (the number of cores on each node of ADA) to compute the
blocks of \verb'C' in parallel.

\end{itemize}

Each of the methods will be test on a single set of square matrices of
size 2048, but with different block sizes for methods 2 and 3.

I have written some of this project for you to get you started.
See the \verb'proj1.zip' file.  The \verb'mult.c' file contains the main
test program and skeletons of the three functions you need to write.
It also provides a helper macro called \verb'SUBMATRIX' that gives you a
simple method of passing submatrices to a function:

\begin{verbatim}
#define SUBMATRIX(C,I,J,b) ((double (*)[NMAX]) & (C [I*b][J*b]))
\end{verbatim}

Suppose \verb'A' is a 2D array of size \verb'NMAX'-by-\verb'NMAX', defined
as:

\begin{verbatim}
double A [NMAX][NMAX] ;
\end{verbatim}

A submatrix can be passed to a function \verb'func' using, for exampe:

\begin{verbatim}
func (SUBMATRIX (A, 2, 5, 8), 8) ;
\end{verbatim}

where \verb'func' is defined as

\begin{verbatim}
void func (double X [ ][NMAX], int n)
\end{verbatim}

This submatrix \verb'X' is the (2,5) block of the matrix \verb'A', of size
8-by-8.  It consists of the region of \verb'A' in rows 16 to 23 (starting
at row 2*8) and columns 40 to 47).  Inside the function \verb'func', this
matrix \verb'X' can be accessed as if it were an \verb'n'-by-\verb'n'
matrix with rows and columns range from 0 to 7 (or 0 to \verb'n-1').
That is, the function \verb'func' does not need to be aware that \verb'X' is
a submatrix of a larger matrix.  All it needs to know is that each row
of \verb'X' is held in a memory space of size \verb'NMAX'.  The C compiler
will handle all the index calculations for you.  For example, the entry
\verb'X[0][0]' is the same as \verb'A[16][40]',
\verb'X[1][2]' is the same as \verb'A[17][42]', and so on.

Files in \verb'proj1.zip':

\begin{itemize}
\item \verb'Makefile':  for both Linux and the Mac OSX.  If you test it on
a Mac, you will need to install XCode, including the command line utilities.
Testing on a Mac is handy, if you have one, but for
best timing results, however, please report your final results on ADA.

\item \verb'mult.c': the primary program.  See the \verb'TODO' comments.
I have written the test driver for you.  All you have to do is write the
three matrix multiplication methods.

\item \verb'tictoc.c': timing routines.

\item \verb'tictoc.h': include file for using \verb'tictoc'.

\item \verb'mymult.lsf':  job submission file to run the tests on ADA.
Use the command {\tt bsub < mymult.lsf}.  You may need to adjust the
run time limit.  To check the status of your jobs use \verb'bjobs'.

\end{itemize}

Write up a short project report discussing your solution and the performance
obtained by each method.  Which method is fastest overall?  What is the peak
speedup over the sequential method (note this is not printed by the 
\verb'mult.c' program)?

Honesty statement:  As Aggies, we follow the Aggie Honor Code.  Period.  This
is a solo assignment.  Do not share code; I can find it no matter how you
obscure it.  Trust me on that; I have been writing code, now used by millions,
since before you were born.  Copying code results in a zero on the project and
a record in the honor system; don't risk it.

\end{document}

